\chapter{Εισαγωγή}
Η ηλεκτρονική αναλυτική επεξεργασία, ή αλλιώς OLAP, επιτρέπει την οργάνωση και συντήρηση μεγάλων βάσεων δεδομένων. Επιτρέπουν ταχύτατες πράξεις για την μείωση διαστάσεων και την σύμπτυξη πληροφορίας. Ωστόσο, αυτό απαιτεί ορισμένη εμπειρία και αναπόφευκτη ώρα εργασίας για να βρεθούν ουσιώδη αποτελέσματα, τα οποία πολλές φορές αφήνονται στην κρίση του χρήστη.


\section{Σκοπός της εργασίας}
Σκοπός της εργασίας είναι να αυτοματοποιηθεί κατά μεγάλο ποσοστό η διεργασία για την εξαγωγή συμπερασμάτων. Ο χρήστης δεν χρειάζεται να κάνει πράξεις, βέβαια πρέπει να ορίσει ορισμένες μεταβλητές όπως τον αριθμό των insights (συμπερασμάτων) που αναζητά, τις διαστάσεις που τον ενδιαφέρουν κτλ. Ο τρόπος για να υλοποιηθεί αυτή η διεργασία είναι:
\begin{itemize}
\item \textbf{Υπολογισμός όλων των δυνατών συνδυασμών extractor (εξαγωγέων).}\\ Οι extractors είναι παρόμοιοι με τις συναρτήσεις. Δέχονται μια πληθώρα μεταβλητών και επιστρέφουν μια τιμή πίσω.

\item \textbf{Υπολογισμός όλων των δυνατών υποχώρων.}\\ Υποχώρος θεωρείται ένας χώρος που ανήκει στο σύνολο των δεδομένων που επεξεργαζόμαστε.

\item \textbf{Εξαγωγή ενός συνόλου τιμών.}\\ Κάθε έγκυρο ζεύγος extractor με υποχώρο δίνει ένα συγκεκριμένο σύνολο τιμών.

\item \textbf{Υπολογισμός σκορ από αυτό το σύνολο.}\\ Ορίζεται μια συνάρτηση όπου επιστρέφει ένα σκορ αναλόγως τον τύπο του insight που αναζητείται.
\end{itemize}

Στην συνέχεια της εργασίας θα αναλυθούν διεξοδικά τα βήματα ένα προς ένα.


\section{Δομή της εργασίας}
Για να γίνει πλήρως αντιληπτό το αντικείμενο που αναλύεται, η δομή της εργασίας ακολουθεί αυτόν τον απλό σκελετό:
\begin{enumerate}

\item Αναφέρονται με πλήρη σαφήνεια όλοι οι ορισμοί και το υπόβαθρο που πρέπει να υπάρχει (Κεφάλαιο 2) και ύστερα αναλύονται πρακτικές χρήσεις που έχουν βρεθεί στον πραγματικό κόσμο. Αναφέρονται επίσης παρόμοιες εργασίες (Κεφάλαιο 3).

\item Ορίζεται ένας βασικός αλγόριθμος σε ψευτοκώδικα (Κεφάλαιο 4) και ύστερα παρουσιάζεται η δικιά μας υλοποίηση (Κεφάλαιο 5).

\item 

\item Κλείνουμε την εργασία με τελικά συμπεράσματα.

\end{enumerate}

