%% Thesis.tex -- MAIN FILE 
%% ---------------------------------------------------------------- 

% Set up the document
\documentclass[a4paper, 11pt, oneside]{Thesis}  % Use the "Thesis" style, based on the ECS Thesis style by Steve Gunn
\graphicspath{Figures/}  % Location of the graphics files (set up for graphics to be in PDF format)
% Table configuration packages
\usepackage{array,graphicx}
\usepackage{booktabs}
\usepackage{pifont}
\usepackage{tabu}
\usepackage{longtable}
\usepackage{xcolor}
\usepackage{tcolorbox}
\usepackage{textcomp}
% Υποστήριξη για ελληνικά
\usepackage[utf8]{inputenc}
\usepackage[greek]{babel}
\usepackage{alphabeta}
\usepackage[LGR, T1]{fontenc}
\usepackage{multicol}

\makeatother

% Include any extra LaTeX packages required
\usepackage[square, numbers, comma, sort&compress]{natbib}  % Use the "Natbib" style for the references in the Bibliography
\usepackage{verbatim}  % Needed for the "comment" environment to make LaTeX comments
\usepackage{float} % To keep figures in place
\hypersetup{urlcolor=black, colorlinks=false, pdfborder = {0 0 0}}  % Colours hyperlinks in blue
% Define enumerated description lists
\usepackage{enumitem}
\newcounter{descriptcount}
\newcounter{descriptcount2}
\newlist{enumdescript}{description}{2}
\setlist[enumdescript,1]{%
  before={\setcounter{descriptcount}{0}%
          \renewcommand*\thedescriptcount{\arabic{descriptcount}.}}
  ,font=\bfseries\stepcounter{descriptcount}\thedescriptcount~
}
\setlist[enumdescript,2]{%
  before={\setcounter{descriptcount2}{0}%
          \renewcommand*\thedescriptcount{\roman{descriptcount2}.}}
  ,font=\bfseries\stepcounter{descriptcount2}\thedescriptcount~
}
 

 
%% ----------------------------------------------------------------
\begin{document}

% For changes in supervisor, degree type, research group, etc. please change the Thesis.cls file
\frontmatter      % Begin the book's numbering; frontpage
%\pagenumbering{arabic}

% Set up the Title Page
\title  {\subjectTitle
}


\authors  {\texorpdfstring
            {\href{mailto:kynigopou@csd.auth.gr}{Γεώργιου Κυνηγόπουλου - 3093}}
            {Όνομα Φοιτητή (ΑΕΜ)}
            }
\addresses  {\groupname\\\deptname\\\univname}  % Do not change this here, instead these must be set in the "Thesis.cls" file, please look through it instead
\date       {25/06/2021}
\subject    {}
\keywords   {}

\maketitle

%% ----------------------------------------------------------------

\setstretch{1.3}  % It is better to have smaller font and larger line spacing than the other way round

% Define the page headers using the FancyHdr package and set up for one-sided printing
\fancyhead{}  % Clears all page headers and footers
\rhead{\thepage}  % Sets the right side header to show the page number
\lhead{}  % Clears the left side page header

\pagestyle{fancy}  % Finally, use the "fancy" page style to implement the FancyHdr headers

%% ----------------------------------------------------------------
% Declaration Page required for the Thesis, your institution may give you a different text to place here
\Declaration{

\addtocontents{toc}{\vspace{1em}}  % Add a gap in the Contents, for aesthetics

Εγώ, ο Γεώργιος Κυνηγόπουλος, δηλώνω ότι αυτή η πτυχιακή εργασία με τίτλο, \subjectTitle, και η δουλειά που παρουσιάζεται σε αυτή είναι δικά μου. Επιβεβαιώνω ότι:

\begin{itemize} 
\item[\tiny{$\blacksquare$}] Αυτή η δουλειά πραγματοποιήθηκε ολοκληρωτικά ή κυρίως κατά την υποψηφιότητά μου για τίτλο προπτυχιακών σπουδών σε αυτό το πανεπιστήμιο.
 
\item[\tiny{$\blacksquare$}] Όπου οποιοδήποτε μέρος αυτής της πτυχιακής εργασίας έχει προηγουμένως κατατεθεί για την απόκτηση πτυχίου ή άλλου τίτλου σε αυτό ή άλλο πανεπιστήμιο, αυτό διατυπώνεται ξεκάθαρα.
 
\item[\tiny{$\blacksquare$}] Όπου έχω συμβουλευτεί την δημοσιευμένη δουλειά τρίτων, αυτό αποδίδεται ορθώς.
 
\item[\tiny{$\blacksquare$}] Όπου έχω παραθέσει από δουλειά τρίτων, η πηγή δίνεται πάντα. Με εξαίρεση αυτές τις παραθέσεις, αυτή η πτυχιακή εργασία είναι εξ ολοκλήρου προσωπική μου δουλειά.
 
\item[\tiny{$\blacksquare$}] Έχω παραθέσει όλες τις κύριες πηγές βοήθειας.
 
\item[\tiny{$\blacksquare$}] Όπου αυτή η πτυχιακή εργασία είναι βασισμένη σε συνεργατική δουλειά δική μου και τρίτων, έχω καταστήσει ξεκάθαρο ποια κομμάτια έχουν πραγματοποιηθεί από άλλους και πώς συνέβαλα εγώ.
\\
\end{itemize}
 
 
Υπογραφή:\\
\rule[1em]{25em}{0.5pt}  % This prints a line for the signature
 
Ημερομηνία:\\
\rule[1em]{25em}{0.5pt}  % This prints a line to write the date
}
\clearpage  % Declaration ended, now start a new page

%% ----------------------------------------------------------------

% The Abstract Page
\addtotoc{Σύνοψη}  % Add the "Abstract" page entry to the Contents
\abstract{
\addtocontents{toc}{\vspace{1em}}  % Add a gap in the Contents, for aesthetics

Με την εκθετική αύξηση της ποσότητας πληροφορίας που μας κατακλύζει, η εξαγωγή insights (συμπερασμάτων) από πολυδιάστατα δεδομένα καθίσταται όλο και πιο δύσκολη, απαιτώντας αρκετή εμπειρία στον τομέα της ανάλυσης δεδομένων. Σε αυτή την εργασία, αναλύεται ο τρόπος για να αυτοματοποιηθεί αυτή η διεργασία, επιτρέποντας ακόμα και σε μη-έμπειρους χρήστες να εξάγουν εύκολα συμπεράσματα σε γρήγορο χρόνο. Επιπλέον, προτείνονται πρακτικές χρήσεις για το συγκεκριμένο λογισμικό, επιπρόσθετες επεκτάσεις που θα μπορούσαν να εφαρμοστούν και παρουσιάζονται αποτελέσματα με χρόνους εκτέλεσης.

}

\clearpage  % Abstract ended, start a new page
%% ----------------------------------------------------------------

\setstretch{1.3}  % Reset the line-spacing to 1.3 for body text (if it has changed)

% The Acknowledgements page, for thanking everyone
\acknowledgements{
\addtocontents{toc}{\vspace{1em}}  % Add a gap in the Contents, for aesthetics

Θα ήθελα να ευχαριστήσω θερμά τον επιβλέποντα καθηγητή, κ. Αναστάσιο Γούναρη για την εξαιρετική επικοινωνία που υπήρχε, για την καθοδήγησή του και για την επίλυση όλων των αποριών κατά την διάρκεια αυτής της πτυχιακής εργασίας.

Επιπλέον, θα ήθελα να ευχαριστήσω την οικογένειά μου, η οποία με στήριξε για την ολοκλήρωση των σπουδών μου.


}
\clearpage  % End of the Acknowledgements
%% ----------------------------------------------------------------

\pagestyle{fancy}  %The page style headers have been "empty" all this time, now use the "fancy" headers as defined before to bring them back


%% ----------------------------------------------------------------
\lhead{\emph{Περιεχόμενα}}  % Set the left side page header to "Contents"
\tableofcontents  % Write out the Table of Contents


%% ----------------------------------------------------------------

\mainmatter	  % Begin normal, numeric (1,2,3...) page numbering
\pagestyle{fancy}  % Return the page headers back to the "fancy" style

% Include the chapters of the thesis, as separate files
% Just uncomment the lines as you write the chapters

\chapter{Εισαγωγή} \label{introduction}
Δοκιμαστικό κείμενο: Έτσι αλλάζει η γλώσσα στα αγγλικά \textlatin{example}.
 % Introduction



%% ----------------------------------------------------------------
% Now begin the Appendices, including them as separate files

\addtocontents{toc}{\vspace{2em}} % Add a gap in the Contents, for aesthetics

\appendix % Cue to tell LaTeX that the following 'chapters' are Appendices

% \chapter{Παράδειγμα Παραρτήματος} \label{math}

 

% Εδώ μπαίνουν τα παραρτήματα

\addtocontents{toc}{\vspace{2em}}  % Add a gap in the Contents, for aesthetics



\end{document}  % The End