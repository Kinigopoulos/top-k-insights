%% Thesis.tex -- MAIN FILE 
%% ---------------------------------------------------------------- 

% Set up the document
\documentclass[a4paper, 11pt, oneside]{Thesis}  % Use the "Thesis" style, based on the ECS Thesis style by Steve Gunn
\graphicspath{Figures/}  % Location of the graphics files (set up for graphics to be in PDF format)
% Table configuration packages
\usepackage{array,graphicx}
\usepackage{booktabs}
\usepackage{pifont}
\usepackage{tabu}
\usepackage{longtable}
\usepackage{xcolor}
\usepackage{tcolorbox}
\usepackage{textcomp}
% Υποστήριξη για ελληνικά
\usepackage[utf8]{inputenc}
\usepackage[greek]{babel}
\usepackage{alphabeta}
\usepackage[LGR, T1]{fontenc}
\usepackage{multicol}

\makeatother

% Include any extra LaTeX packages required
\usepackage[square, numbers, comma, sort&compress]{natbib}  % Use the "Natbib" style for the references in the Bibliography
\usepackage{verbatim}  % Needed for the "comment" environment to make LaTeX comments
\usepackage{float} % To keep figures in place
\hypersetup{urlcolor=black, colorlinks=false, pdfborder = {0 0 0}}  % Colours hyperlinks in blue
% Define enumerated description lists
\usepackage{enumitem}
\newcounter{descriptcount}
\newcounter{descriptcount2}
\newlist{enumdescript}{description}{2}
\setlist[enumdescript,1]{%
  before={\setcounter{descriptcount}{0}%
          \renewcommand*\thedescriptcount{\arabic{descriptcount}.}}
  ,font=\bfseries\stepcounter{descriptcount}\thedescriptcount~
}
\setlist[enumdescript,2]{%
  before={\setcounter{descriptcount2}{0}%
          \renewcommand*\thedescriptcount{\roman{descriptcount2}.}}
  ,font=\bfseries\stepcounter{descriptcount2}\thedescriptcount~
}
 

 
%% ----------------------------------------------------------------
\begin{document}

% For changes in supervisor, degree type, research group, etc. please change the Thesis.cls file
\frontmatter      % Begin the book's numbering; frontpage
%\pagenumbering{arabic}

% Set up the Title Page
\title  {Τίτλος της Διπλωματικής}

\authors  {\texorpdfstring
            {\href{mailto:author@csd.auth.gr}{Ονομα Φοιτητή - ΑΕΜ}}
            {Όνομα Φοιτητή (ΑΕΜ)}
            }
\addresses  {\groupname\\\deptname\\\univname}  % Do not change this here, instead these must be set in the "Thesis.cls" file, please look through it instead
\date       {Ημερομηνία}
\subject    {}
\keywords   {}

\maketitle

%% ----------------------------------------------------------------

\setstretch{1.3}  % It is better to have smaller font and larger line spacing than the other way round

% Define the page headers using the FancyHdr package and set up for one-sided printing
\fancyhead{}  % Clears all page headers and footers
\rhead{\thepage}  % Sets the right side header to show the page number
\lhead{}  % Clears the left side page header

\pagestyle{fancy}  % Finally, use the "fancy" page style to implement the FancyHdr headers

%% ----------------------------------------------------------------
% Declaration Page required for the Thesis, your institution may give you a different text to place here
\Declaration{

\addtocontents{toc}{\vspace{1em}}  % Add a gap in the Contents, for aesthetics

Εγώ, ο/η [όνομα φοιτητή], δηλώνω ότι αυτή η πτυχιακή εργασία με τίτλο, [τίτλος διπλωματικής], και η δουλειά που παρουσιάζεται σε αυτή είναι δικά μου. Επιβεβαιώνω ότι:

\begin{itemize} 
\item[\tiny{$\blacksquare$}] Αυτή η δουλειά πραγματοποιήθηκε ολοκληρωτικά ή κυρίως κατά την υποψηφιότητά μου για τίτλο προπτυχιακών σπουδών σε αυτό το πανεπιστήμιο.
 
\item[\tiny{$\blacksquare$}] Όπου οποιοδήποτε μέρος αυτής της πτυχιακής εργασίας έχει προηγουμένως κατατεθεί για την απόκτηση πτυχίου ή άλλου τίτλου σε αυτό ή άλλο πανεπιστήμιο, αυτό διατυπώνεται ξεκάθαρα.
 
\item[\tiny{$\blacksquare$}] Όπου έχω συμβουλευτεί την δημοσιευμένη δουλειά τρίτων, αυτό αποδίδεται ορθώς.
 
\item[\tiny{$\blacksquare$}] Όπου έχω παραθέσει από δουλειά τρίτων, η πηγή δίνεται πάντα. Με εξαίρεση αυτές τις παραθέσεις, αυτή η πτυχιακή εργασία είναι εξ ολοκλήρου προσωπική μου δουλειά.
 
\item[\tiny{$\blacksquare$}] Έχω παραθέσει όλες τις κύριες πηγές βοήθειας.
 
\item[\tiny{$\blacksquare$}] Όπου αυτή η πτυχιακή εργασία είναι βασισμένη σε συνεργατική δουλειά δική μου και τρίτων, έχω καταστήσει ξεκάθαρο ποια κομμάτια έχουν πραγματοποιηθεί από άλλους και πώς συνέβαλα εγώ.
\\
\end{itemize}
 
 
Υπογραφή:\\
\rule[1em]{25em}{0.5pt}  % This prints a line for the signature
 
Ημερομηνία:\\
\rule[1em]{25em}{0.5pt}  % This prints a line to write the date
}
\clearpage  % Declaration ended, now start a new page

%% ----------------------------------------------------------------
% The "Funny Quote Page"
\pagestyle{empty}  % No headers or footers for the following pages

\null\vfill
% Now comes the "Funny Quote", written in italics
\textit{\textlatin{Απόφθεγμα (προεραιτικό)}}

\begin{flushright}
\textlatin{Συγγραφέας Αποφθέγματος}
\end{flushright}

\vfill\vfill\vfill\vfill\vfill\vfill\null
\clearpage  % Funny Quote page ended, start a new page
%% ----------------------------------------------------------------

% The Abstract Page
\addtotoc{Σύνοψη}  % Add the "Abstract" page entry to the Contents
\abstract{
\addtocontents{toc}{\vspace{1em}}  % Add a gap in the Contents, for aesthetics

Εδώ γράφεται η σύνοψη της δουλειάς.

}

\clearpage  % Abstract ended, start a new page
%% ----------------------------------------------------------------

\setstretch{1.3}  % Reset the line-spacing to 1.3 for body text (if it has changed)

% The Acknowledgements page, for thanking everyone
\acknowledgements{
\addtocontents{toc}{\vspace{1em}}  % Add a gap in the Contents, for aesthetics

Εδώ γράφονται οι ευχαριστίες.


}
\clearpage  % End of the Acknowledgements
%% ----------------------------------------------------------------

\pagestyle{fancy}  %The page style headers have been "empty" all this time, now use the "fancy" headers as defined before to bring them back


%% ----------------------------------------------------------------
\lhead{\emph{Περιεχόμενα}}  % Set the left side page header to "Contents"
\tableofcontents  % Write out the Table of Contents

%% ----------------------------------------------------------------
\lhead{\emph{Κατάλογος Σχημάτων}}  % Set the left side page header to "List if Figures"
\listoffigures  % Write out the List of Figures

%% ----------------------------------------------------------------
\lhead{\emph{Κατάλογος Πινάκων}}  % Set the left side page header to "List of Tables"
\listoftables  % Write out the List of Tables

%% ----------------------------------------------------------------
\setstretch{1.5}  % Set the line spacing to 1.5, this makes the following tables easier to read
\clearpage  % Start a new page
\lhead{\emph{Συντομογραφίες}}  % Set the left side page header to "Abbreviations"
\listofsymbols{ll}  % Include a list of Abbreviations (a table of two columns)
{
% \textbf{Acronym} & \textbf{W}hat (it) \textbf{S}tands \textbf{F}or \\
% Εδώ μπαίνουν οι συντομογραφίες
}

\lhead{}
%% ----------------------------------------------------------------
% End of the pre-able, contents and lists of things
% Begin the Dedication page

\setstretch{1.3}  % Return the line spacing back to 1.3

\pagestyle{empty}  % Page style needs to be empty for this page
\dedicatory{Αφιέρωση (προεραιτική)}

\addtocontents{toc}{\vspace{2em}}  % Add a gap in the Contents, for aesthetics


%% ----------------------------------------------------------------
\mainmatter	  % Begin normal, numeric (1,2,3...) page numbering
\pagestyle{fancy}  % Return the page headers back to the "fancy" style

% Include the chapters of the thesis, as separate files
% Just uncomment the lines as you write the chapters

\chapter{Εισαγωγή}
Η ηλεκτρονική αναλυτική επεξεργασία, ή αλλιώς OLAP, επιτρέπει την οργάνωση και συντήρηση μεγάλων βάσεων δεδομένων. Επιτρέπουν ταχύτατες πράξεις για την μείωση διαστάσεων και την σύμπτυξη πληροφορίας. Ωστόσο, αυτό απαιτεί ορισμένη εμπειρία και αναπόφευκτη ώρα εργασίας για να βρεθούν ουσιώδη αποτελέσματα, τα οποία πολλές φορές αφήνονται στην κρίση του χρήστη.


\section{Σκοπός της εργασίας}
Σκοπός της εργασίας είναι να αυτοματοποιηθεί κατά μεγάλο ποσοστό η διεργασία για την εξαγωγή συμπερασμάτων. Ο χρήστης δεν χρειάζεται να κάνει πράξεις, βέβαια πρέπει να ορίσει ορισμένες μεταβλητές όπως τον αριθμό των insights (συμπερασμάτων) που αναζητά, τις διαστάσεις που τον ενδιαφέρουν κτλ. Ο τρόπος για να υλοποιηθεί αυτή η διεργασία είναι:
\begin{itemize}
\item \textbf{Υπολογισμός όλων των δυνατών συνδυασμών extractor (εξαγωγέων).}\\ Οι extractors είναι παρόμοιοι με τις συναρτήσεις. Δέχονται μια πληθώρα μεταβλητών και επιστρέφουν μια τιμή πίσω.

\item \textbf{Υπολογισμός όλων των δυνατών υποχώρων.}\\ Υποχώρος θεωρείται ένας χώρος που ανήκει στο σύνολο των δεδομένων που επεξεργαζόμαστε.

\item \textbf{Εξαγωγή ενός συνόλου τιμών.}\\ Κάθε έγκυρο ζεύγος extractor με υποχώρο δίνει ένα συγκεκριμένο σύνολο τιμών.

\item \textbf{Υπολογισμός σκορ από αυτό το σύνολο.}\\ Ορίζεται μια συνάρτηση όπου επιστρέφει ένα σκορ αναλόγως τον τύπο του insight που αναζητείται.
\end{itemize}

Στην συνέχεια της εργασίας θα αναλυθούν διεξοδικά τα βήματα ένα προς ένα.


\section{Δομή της εργασίας}
Για να γίνει πλήρως αντιληπτό το αντικείμενο που αναλύεται, η δομή της εργασίας ακολουθεί αυτόν τον απλό σκελετό:
\begin{enumerate}

\item Αναφέρονται με πλήρη σαφήνεια όλοι οι ορισμοί και το υπόβαθρο που πρέπει να υπάρχει (Κεφάλαιο 2) και ύστερα αναλύονται πρακτικές χρήσεις που έχουν βρεθεί στον πραγματικό κόσμο. Αναφέρονται επίσης παρόμοιες εργασίες (Κεφάλαιο 3).

\item Ορίζεται ένας βασικός αλγόριθμος σε ψευτοκώδικα (Κεφάλαιο 4) και ύστερα παρουσιάζεται η δικιά μας υλοποίηση (Κεφάλαιο 5).

\item 

\item Κλείνουμε την εργασία με τελικά συμπεράσματα.

\end{enumerate}

 % Introduction

\chapter{Επισκόπηση Βιβλιογραφίας} \label{literature}

 % Review of the Literature

\chapter{Σημειογραφία \& Βασικές Αρχές} \label{methodology}

 % Fundamentals

\chapter{Αρχές και Πλαίσιο Σχεδίασης για [το πρόβλημα προς επίλυση]} \label{framework}

 % Framework

\chapter{Υλοποίηση και Συστατικά Μέρη της Πλατφόρμας [Τίτλος Πλατφόρμας]} \label{platform}
 % Platform

\chapter{Πειραματισμός \& Αποτελέσματα} \label{experimentationANDresults}

 % Experiment 2

\chapter{Συμπεράσματα \& Προτάσεις για Μελλοντική Έρευνα} \label{conclusions}
https://www.kaggle.com/sudalairajkumar/novel-corona-virus-2019-dataset
https://www.kaggle.com/datasnaek/chess
https://www.kaggle.com/kyanyoga/sample-sales-data % Results and Discussion

\chapter{Υλοποίηση και Συστατικά Μέρη της Πλατφόρμας [Τίτλος Πλατφόρμας]} \label{platform}

%\input{Chapters/Chapter7} % Conclusion

%% ----------------------------------------------------------------
% Now begin the Appendices, including them as separate files

\addtocontents{toc}{\vspace{2em}} % Add a gap in the Contents, for aesthetics

\appendix % Cue to tell LaTeX that the following 'chapters' are Appendices

% \chapter{Παράδειγμα Παραρτήματος} \label{math}

 

% Εδώ μπαίνουν τα παραρτήματα

\addtocontents{toc}{\vspace{2em}}  % Add a gap in the Contents, for aesthetics


%% ----------------------------------------------------------------
\label{Bibliography}
\lhead{\emph{Βιβλιογραφία}}  % Change the left side page header to "Bibliography"
\bibliographystyle{ACM-Reference-Format}
% \bibliographystyle{unsrtnat}  % Use the "unsrtnat" BibTeX style for formatting the Bibliography
\textlatin{
\bibliography{Bibliography}  % The references (bibliography) information are stored in the file named "Bibliography.bib"
}
\end{document}  % The End